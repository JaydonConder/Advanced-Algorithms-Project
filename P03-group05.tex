\documentclass[11pt]{article}
\usepackage[margin=1in]{geometry}

\usepackage{algorithm}
\usepackage{algorithmicx}
\usepackage{algpseudocode}

%opening
\title{Raft Consensus Algorithm}
\author{Group 5 \\ 
\small{Jaydon Conder, William Dittman, Sawyer Payne, AJ Gayler, Seth Kreitinger, Nashea Wiesner}}

\begin{document}

\maketitle

(NOTE: this is just a sample outline, and you are free to deviate from this format).

TODO: brief introduction.

\section{Motivation}
When a cluster of computers exists, it can be difficult to appoint and maintain a common leader. The Raft Consensus Algorithm (or RCA) was designed to solve this problem with simplicity in mind; an alternative algorithm, Paxos, is notoriously difficult to understand and implement. The RCA improves upon Paxos by maintaining its correctness and performance, but is vastly more understandable. 

\section{Algorithm}
Provide pseudocode for the algorithm that your team is investigating.  Provide pseudocode for the algorithm, and be sure to 

\begin{algorithm}\caption{\textsc{Raft Consensus Algorithm: Leader}}
 \begin{algorithmic}[1]
   \State {\bf Input:} nodeList[] \Comment {Network of computers}
	\State $heartbeat ()$
	\State $main()$
 \Function{heartbeat()}{} \Comment{done every 20ms}
   \For{$i \to nodeList.length()$}
   	\State $sendHeartBeat( nodeList[i])$ \Comment{Sends a heartbeat packet to every machine in the} 
   \EndFor	\Comment{cluster that affirms that the current leader is still leader}
   \EndFunction
   \Function{listenForChanges}{}
%   \If {change from leader}
 %  \State{addEntryInLog()}
 %  \EndIf
   \EndFunction
 \end{algorithmic}
\end{algorithm}

\begin{algorithm}\caption{\textsc{Raft Consensus Algorithm: Candidate}}
 \begin{algorithmic}
 \textsc{IncrementTermCounter}() \Comment {Keeps track of most up-to-date leader}
 \textsc{BroadcastElectionPacket()} \Comment{Send nodes an election notice to cause a vote}
 \Function{majorityOfVotesReceived()}{}
 	\State $this.state = leader$ \Comment{Transition to Leader Node}
 \EndFunction\
 \Function{minorityOfVotesReceived()}{}
 	\State $this.state = follower$ \Comment{Transition to Follwer Node}
 \EndFunction
 \end{algorithmic}
\end{algorithm}

\section{Analysis}
Optional: you are welcome to provide a brief proof of correctness or a 
running time analysis of the algorithm.  The difficulty of doing this will, of 
ocurse, depend on your algorithm.

\section{Discussion}
Conclude with a discussion.  Things that you might want to consider to put in 
the discussion (or maybe in their own sections) include: what are some variants 
of this algorithm? Where can this algorithm be applied?  What improvements can 
be made to this algorithm (perhaps commenting on experimental results of the 
original paper)?  Has there been follow-up work (trace the paper's citations in 
google schoolar!)

\end{document}
